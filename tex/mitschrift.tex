\documentclass[10pt]{report}

%% Text-Encoding festlegen. Mit utf8 oder utf8x funktionieren Umlaute wie gewohnt.
%% (mit Bibtex funktioniert nur utf8)
\usepackage[utf8x]{inputenc}

%% Sprachdatei für Trennregeln, Datum-Format und ähnliches festlegen
%\usepackage[german]{babel}  % nötig für Umlaute
\usepackage[ngerman, english]{babel}

%% optimiert das typographische Erscheinungsbild
\usepackage{microtype}

%% erlaubt Listen einfacher zu formatieren (bietet nosep für kompakte Listen)
\usepackage{enumitem}
%% erlaubt hübsche Tabellen über mehrere Seiten, beinhaltet booktabs (\toprule, \midrule, ...)
\usepackage{ctable}
%% ermöglicht farbigen Text ({\color{red} ...})
\usepackage{xcolor} 


\usepackage{mathrsfs}

%% erweiterte Funktionalität für Formeln (Pakete der American Mathematical Society)
\usepackage{amsfonts,amsmath,amsthm,amssymb}
 \numberwithin{equation}{chapter}

%% vordefinierte Einheiten, einfaches Angeben von Einheiten (\SI{8 \pm 1}{cm})
%%   die Unsicherheit soll mit +- abgetrennt werden
\usepackage[separate-uncertainty]{siunitx}
\sisetup{
    range-units = single,       % \SIrange soll die Einheit nur einmal anzeigen
    list-units  = repeat,       % \SIlist soll die Einheit wiederholen
}
%% bei siuntix funktioniert babel leider nicht
%% für englische Dokumente sollten diese Zeilen auskommentiert werden. 
\sisetup{
    range-phrase         = { bis },
    list-final-separator = { und },
%    list-pair-separator  = { und }, % an Uni noch nicht verfügbar
}

%% erlaubt es Bilddateien einzubinden
%% (ctable graphicx intern auch. Trotzdem ist es sinnvoll graphicx expilizt zu laden.
%%  Sonst entstehen schwehr verständliche Fehler, wenn ctable entfernt wird)
\usepackage{graphicx}
%% ermöglicht Bilder und Tabellen am eingegebenen Ort zu platzieren ([H])
\usepackage{float}
%% ermöglicht Unter-Bilder in einer figure-Umgebung
\usepackage{caption}
\captionsetup[figure]{font=footnotesize, labelfont=bf}
\captionsetup[subfigure]{labelformat=parens, labelsep=space, font=small}

\usepackage{subfig}
%% Grafik-Dateien werden in den folgenden Ordnern gesucht
\graphicspath{{img/}}
%% Grafikdateien haben die folgenden Endungen (höchste Priorität zu erst)
\DeclareGraphicsExtensions{.pdf,.png,.jpg}

%% Vertikaler Abstand zwischen Absätzen, Beginn eines Absatzes nicht einrücken
\usepackage{parskip}
% \setlength{\parskip}{0.6em}   % Vertikaler Abstand zwischen Absätzen anpassen 
% \setlength{\parindent}{0em}   % Einrück-Abstand anpassen 

%% zeige Labels im Seitenrand. Dies ist praktisch um Verweise zu kontrollieren
\usepackage[final]{showkeys} % die Option 'final' deaktiviert die Ausgabe von showkeys

\linespread{1.3}	% 1.3

%% Seiten-Layout einstellen
\usepackage[
 a4paper,
 total={13.8cm,21.7cm},          % Breite und Höhe des Inhalt-Bereichs
 top=40mm, left=36mm,        % Ränder oben und links
 headsep=10mm,               % Abstand des unteren Rands der Kopfzeile vom oberen Rand des Inhalts
 footskip=10mm               % Abstand des unteren des Inhalts zum oberen Rand der Fusszeile
% showframe					 % for troubleshooting
]{geometry}

%% Ermöglicht Links im PDF 
%%   sollte möglichst spät in der Präambel geladen werden
\usepackage[
 pdftex,                        % wir verwenden pdftex/pdflatex
 bookmarks=true,                % wir wollen auch im PDF-Reader ein Inhaltsverzeichnis
 bookmarksdepth=3,              % das Inhaltsverzeichnis soll 3 Tiefen enthalten
 colorlinks=true,               % Linktexte sollen Farbig sein 
 linkcolor=black,               % Links innerhalb des Dokuments bleiben schwarz
 citecolor=black,               % Links zu Quellenangaben bleiben ebenfalls schwarz
 urlcolor=black,                 % URL-Linktexte sollen blau dargestellt werden
%  pdfborder={0 0 0}              % Links im PDF erhalten keinen Rahmen, nur nötig wenn colorlinks=false
]{hyperref}

%% Angaben für die PDF-Eigenschaften
\hypersetup{
  pdfauthor = {Kevin Hauser},
  pdftitle = {QM1 - Zusamenfassung},
  pdfsubject = {LaTeX},
  pdfkeywords = {LaTeX, Beispiel}
}


%% definiert \cref: Referenzen mit korrekter Bezeichnung (z.B. "Abbildung 1")
%%   die Nummer alleine ist weiter mittels \ref verfügbar
%% muss NACH 'hyperref' geladen werden
%\usepackage[german]{cleveref}
\usepackage[english, capitalise]{cleveref}


%================  for subfigure


%%%%%%%%%%%%%%%%%  additional libraries

% link: http://tex.stackexchange.com/questions/36524/how-to-put-a-framed-box-around-text-math-environment
\usepackage{mdframed}
\usepackage{lipsum} % for creating dummy text

\mdfdefinestyle{MyFrame}{%
    linecolor=black,
    outerlinewidth=2pt,
    roundcorner=20pt,
    innertopmargin=\baselineskip,
    innerbottommargin=\baselineskip,
%	leftmargin=1cm,
    innerrightmargin=20pt,
    innerleftmargin=20pt,
    backgroundcolor=gray!10!white,
    frametitlerule =true,
%    frametitlerulewidth=0.5pt,
	frametitlerulecolor=black
    }
    

%=======================================================================
% new commands
%=======================================================================

%\newcommand{\Tc}{
%  $T_c$
%}
%\newcommand{\lsco}[1][]{
%  LSCO#1
%}
%
\newcommand*{\NewPage}{
  \newpage\null\thispagestyle{empty}\newpage
}

\newcommand{\refEq}[1]{
  Eq. \ref{#1}
}

\DeclareMathOperator{\sign}{sign}

%========================================================================================================
%========================================================================================================
% Document
%========================================================================================================
%========================================================================================================

%% Angaben für \maketitle
\title{BA - rough sketch}
\author{Kevin Hauser}
% \date{7. Mai 2013}             % ohne Angabe wird das heutig Datum verwendet

\begin{document}
\pagenumbering{roman} %
%=======================================================================
% Title page
%=======================================================================

\begin{titlepage}
    \begin{center}
%        \vspace*{1cm}
       
        \Huge
        \textbf{Mitschrift KOMA}
       
        \vspace{0.5cm}
        \Large

%        \includegraphics[width=0.4\textwidth]{{{img\manipulator.png}}}
 
 		\vspace{3.0cm}
		\begin{figure}[H]
 		\centering
 		\includegraphics[width=0.5\textwidth]{{{img/uzh_logo}}}
		\end{figure} 

       
        \vspace{1cm}
       
        %A thesis presented for the degree of\\
        %Doctor of Philosophy
       
       

       
       
        \vspace{1.5cm}
       
        Department of Physics\\
        University of Zurich\\
        \today
       
    \end{center}
\end{titlepage}


% empty page
\null
\thispagestyle{empty}
\setcounter{page}{2}


%=======================================================================
% Introduction
%=======================================================================


\begin{abstract}



\setcounter{page}{3}
\end{abstract}

\setcounter{page}{4}
\NewPage
\setcounter{page}{5}

\tableofcontents
\thispagestyle{empty} % don't show (roman) page number on titlepage

%=======================================================================
% Charge Order
%=======================================================================

\chapter{Charge Order}


\section{Peierl Transition}


\section{From Causality to Kramer-Kronig relation}


Looking at a causal function $\tilde{\chi}(t)$, we can split it, like every analytical function, in an even $\chi_{even}(t)$ and an odd $\chi_{odd}(t)$ part.


\begin{minipage}{0.45\textwidth}
  \begin{equation*}
    \tilde{\chi}(t) ~~ = ~~ \left\{ \begin{array}{lc} 
      0        &  t < t_0 \\
      \chi(t)  &  t > t_0
    \end{array}\right.
  \end{equation*}
%   \includegraphics[width=0.8\textwidth]{img/uzh_logo}
\end{minipage}
\begin{minipage}{0.1\textwidth}
\end{minipage}
\begin{minipage}{0.45\textwidth}
   \includegraphics[width=0.8\textwidth]{img/uzh_logo}
\end{minipage}

%Every function $\tilde{\chi}(t)$ can be divided in an even $\chi_{even}(t)$ and an odd $\chi_{odd}(t)$ part.

\begin{equation} \label{eq:even_odd}
\tilde{\chi}(t) ~~ = ~~ \frac{\tilde{\chi}(t) + \tilde{\chi}(-t)}{2} + \frac{\tilde{\chi}(t) - \tilde{\chi}(-t)}{2} ~~ = ~~ \chi_{even}(t) + \chi_{odd}(t)
\end{equation}


Multiplying the even part of this function with the signum function yields,

\begin{equation}
  \sign(t) \cdot \chi_{even} ~~ = ~~ \sign(t) \cdot \left\{ \frac{\tilde{\chi}(t)}{2} + \frac{\tilde{\chi}(-t)}{2} \right\} ~~ = ~~~\frac{\tilde{\chi}(t)}{2} - \frac{\tilde{\chi}(t)}{2} ~~ = ~~ \chi_{odd}(t)
\end{equation}


Using this relation to replace $\chi_{odd}(t)$ in \refEq{eq:even_odd}.


\begin{equation}
\tilde{\chi}(t) ~~ = ~~ \chi_{eve} + \chi_{odd} ~~ = ~~ (1 + \sign(t)) \cdot \chi_{even}(t) ~~ = ~~ \sigma(t) \cdot \chi_{even}(t)
\end{equation}


\begin{thebibliography}{9}

%\bibitem{bednorz86}
%  J.G. Bednorz and K.A. Müller,
%  Possible High T\textsubscript{c} Superconductivity in the Ba-La-Cu-O System
%  \textit{Condens. Matter}, \textbf{64}:189-,
%  1986.
%
%\bibitem{schilling93}
%  A. Schilling, M. Cantoni, J.D. Guo and H.R. Ott,
%  Superconductivity above 130 K in the Hg-Ba-Ca-Cu-O system
%  \textit{Nature}, \textbf{363}:56-,
%  1993.
%  
%\bibitem{ray15}
%  Pia J. Ray,
%  \textit{Master Thesis: Structural investigation of La\textsubscript{2-x}Sr\textsubscript{x}CuO\textsubscript{4}},
%  University of Copenhagen,
%  2015.
%
%
%
%\bibitem{lee06}
%  P. A. Lee, \textit{et. al},
%  Doping a Mott Insulator: Physics of high-temperature superconductivity,
%  \textit{Rev. Mod. Phys.}, \textbf{78}:17-,
%  2006.
%
%\bibitem{tokura00}
%  Y. Tokura, N. Nagaosa,
%  Orbital Physics in Transition-Metal Oxides,    
%  \textit{Science}, \textbf{288}:462-468,
%  2000.
%
%\bibitem{pickett89}
%  W. E. Pickett,
%  Electronic structure of the high-temperature oxid superconductors
%  \textit{Rev. Mod. Phys.}, \textbf{61}:433-,
%  1989.
%  
%\bibitem{emery87}
%  V. J. Emery,
%  Theory of High-$T_c$ Superconductivity in Oxides,
%  \textit{Phys. Rev. Lett.}, \textbf{58}:2794-,
%  1987.
%  
%\bibitem{zhang88}
%  F. C. Zhang and T. M. Rice,
%  Effective Hamiltonian for the superconducting Cu oxides,
%  \textit{Phys. Rev. B}, \textbf{37}:3759-,
%  1988.
%
%\bibitem{tanaka06}
%  S. Tanaka,
%  High-Temperature Superconductivity,
%  \textit{Jpn. J. Appl. Phys.}, \textbf{Vol.45, No.12}:9011-9024,
%  2006.
%  
%\bibitem{rhaman15}
%  Md. A. Rhaman \textit{et al.},
%  A Review on Cuprate Based Superconducting Materials Including Characteristics and Applications,
%  \textit{American Journal of Physics and Applications}, \textbf{Vol.3, No.2}:39-56,
%  2015.
%
%\bibitem{jahn37}
%  H. A. Jahn and E. Teller,
%  Stability of Polyatomic Molecules in Degenerate Electronic States,
%  \textit{Proc. R. Soc. London, Ser. A}, \textbf{161}:200-,
%  1937.
%
%\bibitem{damascelli03}
%  A. Damascelli \textit{et. al},
%  Angle-resolved photoemission studies on the cuprate superconductors,
%  \textit{Rev. Mod. Phys.}, \textbf{75}:473-,
%  2003.
%
%
%\bibitem{damascelli04}
%  A. Damascelli,
%  Probing the Electronic Structure of Complex Systems by ARPES,
%  \textit{Physica Scripta}, \textbf{T109}:61-,
%  2004.
%  
%\bibitem{hufner95}
%  S. H\"ufner,
%  Photoelectron Spectroscopy, 2nd Edition,
%  \textit{Springer-Verlag} New York,
%  1995.
%
%\bibitem{nolting7-7}
%  W. Nolting,
%  Viel-Teilchen-Theorie, 7th Edition,
%  \textit{Springer-Verlag} Berlin Heidelberg,
%  2009.
%
%\bibitem{wang12}
%  W.-P. Wang, \textit{et. al},
%  Orbital characters determined from Fermi surface intensity patterns using angle-resolved photoemission spectroscopy,
%  \textit{Physical Review B}, \textbf{85}:214518,
%  2012.
%
%\bibitem{osterwalder_pes}
%  J. Osterwalder,
%  \textit{Notes from electron spectroscopy lecture},
%  UZH.
%  
%\bibitem{matt12}
%  Christian Matt, 
%  \textit{Master Thesis: The Pseudogap phase in high-temperature superconductors},
%  ETH Zurich,
%  2012.
%  
%\bibitem{matt17}
%  Christian Matt, \textit{et. al},
%  High-temperature Superconductivity Restrained by Orbital Hybridisation,
%  \textit{to be published.}

\end{thebibliography}

\end{document}
